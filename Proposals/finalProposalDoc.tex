% !TEX TS-program = pdflatex
% !TEX encoding = UTF-8 Unicode

%%%BEGIN Article customizations
%Set font, font size and char encoding
\documentclass[11pt]{article} 

\usepackage[utf8]{inputenc} 
\usepackage{graphicx}
\usepackage{eurosym}
\usepackage{xcolor}
\usepackage{lastpage}

%%% PAGE DIMENSIONS
\usepackage{geometry} % to change the page dimensions
\geometry{a4paper} 

\usepackage{graphicx} % support the \includegraphics command and options

% \usepackage[parfill]{parskip} % Activate to begin paragraphs with an empty line rather than an indent

%%% PACKAGES
\usepackage{booktabs} % for much better looking tables
\usepackage{array} % for better arrays (eg matrices) in maths
\usepackage{paralist} % very flexible & customisable lists (eg. enumerate/itemize, etc.)
\usepackage{verbatim} % adds environment for commenting out blocks of text & for better verbatim
\usepackage{subfig} % make it possible to include more than one captioned figure/table in a single float
% These packages are all incorporated in the memoir class to one degree or another...

%%% HEADERS & FOOTERS
\usepackage{fancyhdr} % This should be set AFTER setting up the page geometry
\pagestyle{fancy} % options: empty , plain , fancy
\renewcommand{\headrulewidth}{0pt} % customise the layout...


%%% SECTION TITLE APPEARANCE
\usepackage{sectsty}
\allsectionsfont{\sffamily\mdseries\upshape} 
% (See the fntguide.pdf for font help)
% (This matches ConTeXt defaults)

%%% ToC (table of contents) APPEARANCE
\usepackage[nottoc,notlof,notlot]{tocbibind} % Put the bibliography in the ToC
\usepackage[titles,subfigure]{tocloft} % Alter the style of the Table of Contents
\renewcommand{\cftsecfont}{\rmfamily\mdseries\upshape}
\renewcommand{\cftsecpagefont}{\rmfamily\mdseries\upshape} % No bold!
\usepackage{hyperref}
\hypersetup{
    colorlinks,
    linkcolor={red!50!black},
    citecolor={blue!50!black},
    urlcolor={blue!80!black}
}



\title{\Huge Internship Project Proposal: Procedural avatar generation}
\author{Steven Schenk \& Robert Kraaijeveld \& Cees-Jan Nolen\\
	Institute for Communication, Media and Information technologies\\
	Hogeschool Rotterdam
	Rotterdam, Zuid-Holland, The Netherlands \\
	}
\begin{document}
\nocite{*}
\maketitle
\lhead{Internship Project Proposal: Procedural avatar generation}
\rhead{}
\lfoot{}
\cfoot{}
\rfoot{\thepage\ of \pageref*{LastPage}}

\newpage
\tableofcontents
\pagebreak

\newpage
\phantomsection
\addcontentsline{toc}{section}{Introduction}
\section*{Introduction}
Originally, we wanted to create an 'escape the room'-style puzzle game for Tilburg University, in which a group of players solve puzzles together to 'escape' within a time limit. However, our coordinator Dr. Spronck informed us that another university team was already developing a very similar VR-application.

~\\
We could have collaborated with them or found some other way of implementing our original concept, but we wanted to avoid any overlap with the aforementioned project; therefore, we decided to create a different application altogether.

~\\
Our new concept consists of a tool allowing for the easy creation of in-game avatars with procedurally generated personalities within Unity. If we have enough time left, we are going to incorporate this principle into a restaurant-style 1 player game within the experience room.

~\\
Procedural generation is widely used in the games industry, from fully procedurally generated games like Spore and Minecraft to games like GTA 5 and the recently released No Man's Sky that use procedural generation only for certain tasks like character creation.


\newpage
\phantomsection
\addcontentsline{toc}{section}{Outline of the procedural avatar generation tool/application}
\section*{Outline of the procedural avatar generation tool/application}
The project that we intend to implement will be divided into two parts. 

~\\
The first part consists of a tool, or set of tools, designed to allow Unity developers to easily generate character avatars with distinct bodies and personalities. Giving the developers an easy way of creating lots and lots of engaging avatars for their games.

~\\
Having life-like avatars is a huge factor in player immersion. Especially in games where there is a lot of direct player to avatar interaction, like job simulators and open world games, having avatars walk, talk and act in distinctive and lifelike ways is very important for creating an immersive experience. 
REFERENCE

~\\
Therefore, we want to focus our attention to the procedural generation of character personality. Procedural avatar generation is, like we mentioned earlier, used widely in the creation of games. However, the use of personality generation is less widespread. (With a few exceptions, such as Crusader Kings II REFERENCE). When used, however, it has the potential to greatly increase the players immersion in a video game by increasing replayability. More on this in the next chapter.

~\\
For such a tool to be useful to developers, it would have to be easily usable within development environments. Since we are focusing on Unity with this project, easy integration into Unity is a must-have for this tool.

~\\
The second part of this project, which we consider partially optional, is to prove the worth of our tool within an actual game. More specifically, a restaurant-game set within the Experience Room.

~\\
This game will let the players digitally prepare and serve food for our procedurally generated avatars. Having avatars with different distinct, noticeable personalities and characteristics will both make the game feel more realistic as well as provide additional game-elements. For instance, an avatar with a certain personality might throw a fit when his food is delivered late, leading to other avatars leaving or players being delayed.

~\\
This would be a great way to demonstrate the power of procedurally generated avatars, if we have enough time.

\newpage
\phantomsection
\addcontentsline{toc}{section}{Benefit of creating this application}
\subsection*{Benefit of creating this application}
The obvious benefit to developers that this tool will deliver is enabling them to create a lot of content in a short time. Especially for large, open world games or other types of games with huge content-bases time is of the essence, and not every asset can be created manually.

~\\
A boon for players would be increased replayability. Having NPC's with different personalities and therefore, different decision-making processes populate a gameworld will ensure that every replay is different from the last one. The process of making games more replayable by changing variables upon each playthrough is known as randomization; randomization's value and the importance of replayability has been demonstrated in various both academic research projects such as [1] and [2].

~\\
This replayability-component could also provide an incentive for community activity, as illustrated with the earlier mentioned Crusader Kings II. In this game, players control a medieval dynasty, controlling one reigning family-member at a time. 

~\\
These characters are procedurally generated and the different personality traits of all the player- and non-player characters combined can lead to incredibly complex situations. The complexity of the game itself, combined with the endless amount of situations that can occur have sparked a lot of forum discussions and community-building actions by players. Our tool, if it is to see the light of day, could empower developers wanting to create similar avatar-driven games.

~\\
We do however, think that the use of this tool would be most valuable within the context of open world games, or other types of games players have a large amount of interaction with npc's. Games that have little player-npc interaction or more accurately, social interaction \footnote{For instance, shooting enemy soldier npc's is technically interaction, but certainly not social interaction.}, will benefit less from using this tool since the player will simply not notice the NPC's behaviour; he or she will most likely be to busy performing other tasks.


\newpage
\phantomsection
\addcontentsline{toc}{section}{Undecided issues and limitations}
\subsection*{Undecided issues and limitations}
A possible risk might be that we run of time with this project before we are able to prove our concept by creating a demo-game. Our internship will only last for approx. 6 months and we are new to procedural avatar generation, so it's likely that we will spend the majority of our time working on our procedural generation tool; Leaving little time for the creation of an implementation making use of the tool.

newpage
\phantomsection
\addcontentsline{toc}{section}{Budget plan}
\subsection*{Budget plan}
Finally, we will propose what we consider to be an adequate budget for this project. Because we want to learn as much as possible during this program, we will only be spending money on goods which would take us to long to create for ourselves; art assets. 

~\\
Specifically, we want to purchase two 3d model packs \footnote{These are the 'Real People: Males' and 'Real People: Females' packs, available at the Unity Asset Store.}, consisting of approx. 60 different models each. These will come in the shape of both Unity prefabs and materials, so we can easily customize their texture colors and edit for example, faces, as we choose. They are also compatible with Unity's mecanim system, meaning we will easily be able to create and edit our own animations and apply them to these character models.

~\\
These two packs combined will cost 198 US Dollar.

We decided not to spend any money on hardware like Kinects for use in the experience room, as we are unsure whether an implementation for the Experience Room will actually be created. Even if it would be created, it would likely take the shape of a simple singe-player demo due to the aforementioned time constraint. 


\newpage
\phantomsection
\addcontentsline{toc}{section}{References}
\section*{References}
\begin{}
\item{ [1] T. Frattesi, D. Griesbach, J. Leith, T. Shaffer, and A. Jennifer, "Replayability of video games," 2011. [Online]. Available: http://www.wpi.edu/Pubs/E-project/Available/E-project-051711-130604/unrestricted/Replayability\_of\_Video\_Games\_2011.pdf. Accessed: Aug. 19, 2016.}

\item{
[2]	M. Van Lent, M. O. Riedl, P. Carpenter, R. Mcalinden, and P. Brobst, "Increasing Replayability with deliberative and Reactive planning," 2005. [Online]. Available: http://www.aaai.org/Papers/AIIDE/2005/AIIDE05-023.pdf. Accessed: Aug. 19, 2016.	}
\end{enumerate}



\end{document}