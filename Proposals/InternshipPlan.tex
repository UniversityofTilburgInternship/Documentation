% !TEX TS-program = pdflatex
% !TEX encoding = UTF-8 Unicode

%%%BEGIN Article customizations
%Set font, font size and char encoding
\documentclass[11pt]{article} 

\usepackage[utf8]{inputenc} 
\usepackage{graphicx}
\usepackage{eurosym}
\usepackage{xcolor}
\usepackage{lastpage}
\usepackage{tabularx}

%%% PAGE DIMENSIONS
\usepackage{geometry} % to change the page dimensions
\geometry{a4paper} 

\usepackage{graphicx} % support the \includegraphics command and options

% \usepackage[parfill]{parskip} % Activate to begin paragraphs with an empty line rather than an indent

%%% PACKAGES
\usepackage{booktabs} % for much better looking tables
\usepackage{array} % for better arrays (eg matrices) in maths
\usepackage{paralist} % very flexible & customisable lists (eg. enumerate/itemize, etc.)
\usepackage{verbatim} % adds environment for commenting out blocks of text & for better verbatim
\usepackage{subfig} % make it possible to include more than one captioned figure/table in a single float
% These packages are all incorporated in the memoir class to one degree or another...

%%% HEADERS & FOOTERS
\usepackage{fancyhdr} % This should be set AFTER setting up the page geometry
\pagestyle{fancy} % options: empty , plain , fancy
\renewcommand{\headrulewidth}{0pt} % customise the layout...


%%% SECTION TITLE APPEARANCE
\usepackage{sectsty}
\allsectionsfont{\sffamily\mdseries\upshape} 
% (See the fntguide.pdf for font help)
% (This matches ConTeXt defaults)

%%% ToC (table of contents) APPEARANCE
\usepackage[nottoc,notlof,notlot]{tocbibind} % Put the bibliography in the ToC
\usepackage[titles,subfigure]{tocloft} % Alter the style of the Table of Contents
\renewcommand{\cftsecfont}{\rmfamily\mdseries\upshape}
\renewcommand{\cftsecpagefont}{\rmfamily\mdseries\upshape} % No bold!
\usepackage{hyperref}
\hypersetup{
    colorlinks,
    linkcolor={red!50!black},
    citecolor={blue!50!black},
    urlcolor={blue!80!black}
}


%➢ naam en studentnummer; 
%➢ naam stagebedrijf en bedrijfsbegeleider; 
%➢ start- en einddatum van de stage; 
%➢ omschrijving van het stagebedrijf; 
%➢ doel van de stage;
% ➢ stageopdracht/ stagewerkzaamheden; 
%➢ persoonlijke leerdoelen/ verwachtingen; 
%➢ inschatting van de te behalen competenties (zie bijlage 1).


\title{\Huge Internship plan}
\author{Robert Kraaijeveld 
	Institute for Communication, Media and Information technologies\\
	Hogeschool Rotterdam
	Rotterdam, Zuid-Holland, The Netherlands \\
	}
\begin{document}
\nocite{*}
\date{}
\maketitle
\lhead{Internship plan}
\rhead{}
\lfoot{}
\cfoot{}
\rfoot{\thepage\ of \pageref*{LastPage}}

\newpage
\tableofcontents
\pagebreak

\newpage
\phantomsection
\addcontentsline{toc}{section}{Introduction}
\section*{Introduction}
In this document, I will outline my soon to be started internship at Tilburg University.
To begin, I will list some of the basic information about this internship below.

~\\
• Name and studentnumber: Robert Kraaijeveld - 0890289
~\\
• Internship company and company-coordinator: Tilburg University, dr.ir. P.H.M.
Spronck
~\\
• Starting date and end date of the internship: 5-9-16 until approx. 26-12-2016

Update 25-9-16: This document has been updated to reflect some of my changing thoughts about this internship; specifically, how changes in how I intend to fullfill my competences.

\newpage
\section{The internship project}
We (me and my 2 teammates, supervised by mr. Abbadi and Dr. Spronck) will be
creating a tool allowing for the procedural generation of game avatars with distinct
personalities in Unity. If we have sufficient time, we will implement these avatars within
the context of a restaurant game, played within the Experience Room.

~\\
My internship company/institution will be Tilburg University; specifically, the Department
of Communication and Information Sciences within the Tilburg School of Humanities.
This department of Tilburg University, which employs over a 100 staff-members,
focuses on research and education regarding a wide variety of subjects regarding IT and
digital communication.

~\\
For me personally, this internship will allow me to gain considerable amounts of technical
prowess and allow me to work with state-of-the-art, complex hardware and software
under the supervision of skilled professionals. This may sound like a bit of an advertising
blurb, so let me elaborate.

~\\
The project will be carried out by me and my teammates Cees-Jan Nolen and Steven
Schenk, assisted and supervised by mr. Abbadi and Dr. Spronck. Therefore, this project
is very much ’ours’. I will be able to design, create and deliver an entire project from
the ground up with my teammates and supervisors, rather than making a relatively tiny
contribution to, say, a huge codebase that has been around for 30 years. I think that
this will enable me to learn much more, both from the hands-on experiences that I will
have as well as from my very knowledgeable colleagues/supervisors.

~\\
The second great thing about this internship is the fact that me and my teammates will
be able to work with, and add to, very interesting and very new hardware and software.
The programming language that we are going to use as a ’proxy’ to the visual interface
of Unity is Casanova, a newly created language by (amongst others) mr. Abaddi.

~\\
In the same vein, the aforementioned Experience Room, a 3d mixed reality room located at
Tilburg University, has only been operating for less than a year. Thus our usage of both
of these pieces of hardware and software will provide very useful data on what works and
what doesn’t work with them. This is especially useful for the very young experience
room, being itself an implementation of the very young concept of virtual and mixed
reality.

\newpage
\section{Competences}
Since the Hogeschool Rotterdam requires my internship to facilitate me working on the
6 competences, I have listed how I am going to work on each aspect of these competences
during my internship.

~\\
The six competences are:
~\\
1. Management
~\\
2. Analysing
~\\
3. Consulting
~\\
4. Designing
~\\
5. Implementing
~\\
6. (Social) Skills

~\\
Each competence has several sub-competences associated with it. For each sub-competence
I will answer the following question: ”How will I show that I meet the sub-competence?”.

~\\
An exception being the (social) skills competence, whose sub-competences are only answerable
after I have started my internship.

\newpage
\begin{tabularx}{400pt}{|l|X|X|X|}
\hline No. & Sub-competence & Answer \\
\hline 1 & I can work according to
a pre-made and approved
internship-plan, and provide
motivation for any changes. & I will work in this project using Kanban; a scrum-like
project methodology. Kanban emphasizes working on
small tasks once at a time within small time cycles.
At the same time, kanban still has the concept of a
product backlog, allowing me to use a pre-made task
list. Update: Regularly visiting with both dr. Spronck and mr. Abbadi has often saved us from implementing something that was actually not 100\% in line with the clients wishes. Also, the requirements for the project proved to be much more volatile than I thought.\\
\hline 2 &  I can create an analysis of
the given internship task,
using existing methods and
techniques. Also, I am
able to create a requirementanalysis
for (part of) a software
system with different
stakeholder, whilst taking into
account quality standards. & I will be researching established literature on procedural
avatar generation in order to avoid any duplication
with our research project, since creating a duplicate
tool would not be useful for my stakeholders.
Using tools such as UML and DFD’s I will document
the multiple stakeholders that this project will revolve
around. Update: Carefully documenting our progress and documenting proposed changes proved to be very valuable when communicating with our client. Also, carefull research with the guidance of mr. Abbadi into (for example) possible algorithms for us to use was very interesting and valuable\\
\hline 2 & I am able to create a specifi-
cation using an analysis. & I will be especially targeting this competence during
the later part of the project, when we will create an
VR-implementation of our work. In this stage it will
be of the essence that I implement a specification that
matches the (unexplored) capabilities of the experience
room; since mis-designing the implementation
would mean we end up with a game that does not
match the experience room’s intented use. \\ \hline
\end{tabularx}

\newpage
\begin{tabularx}{400pt}{|l|X|X|X|}
\hline 2 & I am able to create an
acceptance-test using quality
standards. & The very purpose of procedurally generating characters
with distinct personalities is to make games more
immersive for players; it would be very easy to develop
some simple tests to see whether players can ’see the
difference’. Thus, even if me and my teammates would
not be able to develop a fully fledged game, we could
still easily acceptance-test the project. \\
\hline 3 & I am able to provide a wellargumented
and guiding advice
regarding processes, software
and/or new technologies,
and I am able to present this
advice in a convincing and understandable
way. & A large part of my project, and a very important aspect
for Tilburg University is gaining insight into the
potential usage of both the Casanova language and the
Experience Room. Therefore, I will use the results of
the aforementioned acceptance testing as well as my
practical experiences with Casanova to document possible
improvements to the University, or suggest certain
approaches over others. Update: See the update at the 2nd competence.  \\
\hline 4 &I am able to create a design
for (part of a) softwaresystem,
using existing components,
libraries and designqualitystandards. & I will be heavily reliant on existing work during this
project; we will be using the Unity engine as our
’heavy lifter’ and the earlier mentioned Casanova language
as an intermediate language. Also, we will likely
be using external textures and 3D models. Both when
designing the system and during its’ implementation
me and my teammates will be using UML and UMLlike
techniques to communicate both amongst ourselves
and to our client. ERD’s will likely be used
less, unless we are going to be collecting player data. Update: Casanova was often difficult to use because of its experimental nature, but we did learn a lot so far and were able to use mr. Abbadi's knowledge of the language to our advantage.\\
\hline 4 & I am able to validate a design
based on specifications which
resulted from analysis. & I will use the Kanban project methodology combined
with some kind of iteration mechanic. This
will allow me and my teammates to constantly reaffirm
whether what we have implemented matches
our clients’ wishes. Update: See the update at the 2nd competence. \\ \hline
\end{tabularx}


\newpage
\begin{tabularx}{400pt}{|l|X|X|X|}
\hline 5 & I am able to implement software
which conforms the requirements
of the given assignment,
holding this software
by the high standards of
quality that are used in the
software engineering industry
today. & I will be creating C classes following the proxy design
pattern, allowing for these classes to be a wall of
abstraction between the Unity engine and Casanova
code. Casanova therefore only has to deal with relatively
abstract concepts as defined in these proxy
classes, rather than making direct use of the very specific
Unity engine code. The casanova language also
mixes functional and OO-style code, allowing for easy
integration between the proxies and the pure casanova
code. I will also be documenting the code (most likely
in the form of a github wiki), both to allow current and
later developers to easily dive into the project. Another
big reason for documenting is to provide Tilburg
University with useful data on Casanova’s and the Experience
Room performance for this particular task. Update:  My expectation as laid out here originally proved to be true: Casanova best lends itself to very abstract logic-handling rather than more specific implementation details since it doesn't contain constructs like functions etc.\\
\hline 5 & I am able to use unittests,
integrationtests and systemtests.
I am also able to
automate these tests. & I will be, together with my teammates, creating a simple
testing framework for casanova in order to make
testing easier for us. We will also setup a Jenkins
continuous integration service on a virtual server, allowing
us to test automatically whenever we push code
upstream. \\ \hline
\end{tabularx}


\end{document}