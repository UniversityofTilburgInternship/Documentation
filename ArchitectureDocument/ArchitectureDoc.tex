\documentclass{article}

\title{Architecture document}
\date{8-09-2016}
<<<<<<< HEAD
\author{Robert Kraaijeveld}
=======
\author{Robert Kraaijeveld, Cees-Jan Nolen and Steven Schenk}
>>>>>>> eb341849912d01c83266193940db1db6dab3c6b8

\usepackage{graphicx}
\usepackage{fancyhdr}
\usepackage{parskip}
<<<<<<< HEAD
=======
\usepackage[]{algorithm2e}
>>>>>>> eb341849912d01c83266193940db1db6dab3c6b8

\pagestyle{fancy}
\fancyhead[L]{}


\begin{document}
  \pagenumbering{gobble}
  \maketitle
  \newpage
  \pagenumbering{arabic}
  \tableofcontents

  \newpage
  \section{Introduction}
  In this document, we will give an abstract outline of the structure and architecture of our internship project. 
  This will help us provide a more stable vision for the future of this project and will allow us to divide tasks between ourselves more easily.
  Since we cannot yet predict the exact form that this project will take, the architecture will be presented at quite a high level. We will also
  provide an an abstract view of the algorithms that we intend to use for the procedural generation of avatars with distinct personalities.


  \newpage
  \section{ERD Diagram}
  This very basic ERD Diagram describes the way in which we will separate the Unity-specific code, like providing proxy-properties to Casanova from the abstract Casanova code. In this way, the Casanova code for this project will be completely re-usable within a different framework like XNA or Unreal; only the proxies and the XML writer/reader code will have to be altered or rewritten.

  \begin{figure}[h]
	\includegraphics[width=1.4\textwidth]{ER.png}
	\caption{ER-Diagram}
	\label{fig:figure2}
  \end{figure}

  ~\\
  We will most likely provide the big-5 values (or other personality values, more on this later) through an XML file which can be manipulated through the Unity directly inspector. By using an XML file we will be easily able to bring certain personality configurations along on demos and manipulate these configurations from outside of Unity. An important point for us to consider is that the Unity proxies and all other Unity-related pieces of code should not deal with the actual logic of the game; They should only provide an easy interface for Casanova to interact with. By doing this our project will remain relatively independent of Unity, making it easier to use for possible future developers.


  \newpage
  \section{Algorithm structure for a single character}
  In the stages of this project, we will be focusing our attention on having a certain character behave according to his/her personality values and having this character interact with the player accordingly. Characters interacting with other characters is a much more complex subject; we will discuss this in a later chapter. 

  We have identified the most important visible behaviour-characteristics that each avatar should exhibit (almost) regardless of the context of the game in question. There are some exceptions to this rule however, depending on the nature of the game in question. In a relatively static game in which neither player nor npc move around much, like an interrogation-game, movement patterns and walking styles will naturally not be visible. The following list will include examples of each kind of behavior in order to make their distinctions more clear.

  \begin{enumerate}
  	\item Decision-making (Will the character tend to the other NPC who has just fallen over or will I move away?)
  	\item Movement pattern and walking style (Does the NPC move at a steady pace or does he constantly change his walking speed?
  											  Does he walk about confidently or does he slowly wander?) 
  	\item Body Language (What kind of mood does the NPC's body language reflect? Is he standing up proudly or are his shoulders collapsed?)
  	\item Speech (What does the NPC say and with what tone of voice does he say it? What mood does his voice reflect?)
  	\item Facial Expression (Which emotions does the NPC's face reflect? Does his facial expression stay neutral and constant or does it change often and rapidly?)
  \end{enumerate}

  Each of these factors can provide a reasonably visible indication to the player of what kind of personality/emotions the character exhibits. 
  To determine how a certain NPC will display these factors (IE. What the NPC's walking style will be) we will use a personality model. Personality models provide us with a measure of what kind of behavior an NPC who leans to certain sides of a psychological trait would exhibit. 


  \newpage
  \subsection{Change in the behavior of an NPC}
  The behavioral factors of an NPC are changeable and can potentially change for two reasons: Because of the personality of the character itself, or because of certain events that happen to or near the character combined with that characters' personality. Certain personality traits contained in the earlier mentioned personality models, like neuroticism, can cause people to have very variable emotions and extreme emotional reactions to very insignificant events. Within the game, we will likely represent this by having characters with a high neuroticism level change their movement style, body language and facial expression constantly.

<<<<<<< HEAD
  The other way in which we will let our characters behavorial factors change is by subjecting them to certain positive or negative events. Depending on the characters personality, they will respond differently to certain events. An example; NPC 1, who has an egoism value of 100 out of 100 stands next to NPC 2. All of a sudden, NPC 2 lets out a cry of pain and collapses. Because NPC 1 has such a high egoism value, he does not rush to NPC 2's aid: He completely ignores him instead and calmly walks away (An even more cynical reaction might be that NPC 1 starts searching NPC 2 for valuables..).

  If NPC 1 had had a different egoism value, he/she would have reacted differently to this distressing situation. Positive situations will also elicit a certain response depending on the personality values of an NPC. When dance music suddenly starts playing next to a group of NPCs, the extravert ones among them might start to dance, whilst the NPC with very low extraversion scores probably won't dance. 

  Some of the behavorial factors of the NPC's will have to be slightly exagarrated in order for them to be clearly visible to the player and in order for them to be technically viable. Facial expressions in real human beings can be incredible subtle and complex, but emulating exact facial expressions would not only be an incredible technical challenge; it might take extreme player effort in order to actually notice them. Since our focus is still on creating a game, rather than a person simulator, we have prioritized 'playability' over realism.
=======
  The other way in which we will let our characters behavioral factors change is by subjecting them to certain positive or negative events. Depending on the characters personality, they will respond differently to certain events. An example; NPC 1, who has an egoism value of 100 out of 100 stands next to NPC 2. All of a sudden, NPC 2 lets out a cry of pain and collapses. Because NPC 1 has such a high egoism value, he does not rush to NPC 2's aid: He completely ignores him instead and calmly walks away (An even more cynical reaction might be that NPC 1 starts searching NPC 2 for valuables..).

  If NPC 1 had had a different egoism value, he/she would have reacted differently to this distressing situation. Positive situations will also elicit a certain response depending on the personality values of an NPC. When dance music suddenly starts playing next to a group of NPCs, the extravert ones among them might start to dance, whilst the NPC with very low extraversion scores probably won't dance. 

  Some of the behavioral factors of the NPC's will have to be slightly exagarrated in order for them to be clearly visible to the player and in order for them to be technically viable. Facial expressions in real human beings can be incredible subtle and complex, but emulating exact facial expressions would not only be an incredible technical challenge; it might take extreme player effort in order to actually notice them. Since our focus is still on creating a game, rather than a person simulator, we have prioritized 'playability' over realism.
>>>>>>> eb341849912d01c83266193940db1db6dab3c6b8


  \newpage
  \subsection{Our personality models of choice}
<<<<<<< HEAD
=======
   %UITBREIDEM
>>>>>>> eb341849912d01c83266193940db1db6dab3c6b8
   For our first prototype of a single avatar, we will be using the Big 5 personality model. This model has several layers of complexity, but for our very first prototype we will be focusing solely on the top level of the Big 5 model. This personality model containts, as the name implies, 5 principle traits by which to measure personality. They are as follows:

   \begin{itemize}
   	\item Extraversion vs. Introversion
   	\item Egoism vs. Altruism
   	\item Thoroughness vs. Chaoticness \footnote{In the original model, the term 'conscientiousness' is used. Since this term did not seem very descriptive to us, we replaced it with 'Thoroughness'.}
   	\item Emotional Stability vs. Neuroticism 
<<<<<<< HEAD
   	\item Openness to new experiences vs . Stubborness \footnote{Item no. 5 is a little harder to implement as very concrete NPC behavior due to it's relevance lying mostly within the psychological growth and gathering of experiences, therefore we will probably not yet be using much of it in our early prototypes.}
   \end{itemize}

    ~\\
    We have mapped the rest of the factors to the behavioral factors which they will influence:

    ~\\
	\centering
	\label{my-label}
		\begin{tabular}{| p{4cm} | p{4cm} |}
		 \textbf{Behavioral Factor} & \textbf{Big-5 Factor}   \\  \hline
		 Body Language, Speech & Extraversion vs. Intraversion \\  \hline
		 Decision-making & Egoism vs. Altruism    \\  \hline
		 Movement Style & Thoroughness vs. Chaoticness    \\  \hline
 		 Facial Expression & Emotional Stability vs. Neuroticism    
		\end{tabular}
	
	Thusly, the character generation algorithm which we will define will know which personality factor affects the likelihood of a certain behavior factor being exhibited. Next up, we will discuss the actual procedural generation algorithm.

  \newpage
  \subsection{Algorithm structure}
   The algorithm will take as input a list of integers ranging between -100 and 100. Each integer in the list will represent the 'score' for a particular 
   personality factor as defined in the Big five personality model. 

   The score will represent how much the given NPC will exhibit a particular trait; the negative values representing the left side of the two extremes of a particular trait, the positive values representing the right side.

   Given a certain score, the algorithm will take the following (abstract) steps:

   \begin{enum}
   \item Divide the given score by 2. 
   \item If the resulting number is negative, mark this NPC as possessing the left handed trait of the current dicotomy (For instance, Egoism out of 'Egoism vs. Altruism'). Else, mark the NPC as possessing the right handed trait of the current dicotomy.
   \item Whenever the relevant trait of the NPC is triggered\footnote{As mentioned earlier, this can happen due to a positive or negative event or due to the value of a certain trait itself.} use the calculated number to determine how mild or severe the response will be.
   \end{enum}

   It is worth noting that we haven't determined exactly how the mildness or severeness of the NPC response will be turned into actual behavior, but there are a number of options. A simple measure would be to have the calculated number act as a multiplicant for an animation speed variable, or to use it as the duration for a certain animation or state. Slightly more complex measures would be things like having the value of the number decide on a certain animation, walking pattern or action. For instance, the number for egoism could act as the deciding factor regarding wether the earlier mentioned NPC 1 will rob the collapsed NPC 2 or wether he will perform a slightly less egoistic action.

   A potential challenge will be to decide what kind of personality factor is triggered by which kind of external event. There are some no-brainer events of course, like a fellow NPC collapsing on the floor: This event is undoubtedly most relatable to a persons' egoism vs. altruism trait. An alarm that suddenly starts will likely trigger the emotional stability trait of a character, provoking either a mild or severe response from his/her end of the emotional stability vs. neuroticism trait. 

   Other scenarios are much more ambiguous however: The police arriving to arrest an NPC could provoke a lot of different responses: Altruism for the arrestee, Neuroticism at the sight of the police arriving, an extravert response to a particular NPC being arrested, and so on. 
   %DIT UITBREIDDEN


   \newpage
   \section{Differentiating behavioral factor to personality factor bindings}

  %Algorithm steps, inc hoe we de waardes aan persoonlijkheid geven
  %Beschrijving van big5 model en andere modellen; Beginnen met basic big5, dan uitgebreide big5 met facetten, dan hexaco etc.
  %Interactie tussen NPCs
  %'Psychopathietheorie' uitleggen; '' en personality factors zijn niet altijd 1 op 1 het zelfde gekoppeld voor ieder persoon
  %Hoe gaat de demo eruitzien, welke events kiezen we uit
=======
   	\item Openness to new experiences vs . Stubborness (Not supported) \footnote{Item no. 5 is a little harder to implement as very concrete NPC behavior due to it's relevance lying mostly within the psychological growth and gathering of experiences, therefore we will not be using much it in our prototypes.}
   \end{itemize}

    Each of the left-handed personality factors will be given percentage, ranging from 0 to 100 percent. A 100 percent score will mean that the NPC will totally exhibit the given personality 	   factor of the Big 5 model, whilst a 0 percent score will indicate that the NPC will totally exhibit the opposite factor (In the case of Extraversion, this would mean that the NPC is totally                   introvert). In order to allow the user to create populations of NPC's without having to set each of their values individually, each personality factor will have both a minimum and a maximum  percentage. These percentages will indicate between which percentages the scores for this personality factors will range for the NPC population. 

  In the future, we will likely replace the Big-5 model by more complex and larger personality models like HEXACO, in order to create more nuanced and refined NPC personalities. This would also require creating more distinct behaviors in order to accomodate the larger amount of personality factors contained within, for instance, HEXACO.

   \newpage
   \subsection{Algorithm structure}
    The algorithm which we will describe here will be responsible for deciding for each character what kind of behavior he/she will display in a given situation. This behavior will consist of a combination of facial expression, body language, speech \footnote{Since we will most likely not have professional voice actors at our disposal, speech will probably consist of text bubbles or even just simple emotes like exclamation marks etc.} movement style and movement pattern and certain decision-making processes. 

    Before the actual algorithm can make decisions on an NPC's behavior, the following will need to be defined:

    \begin{enumerate}
    	\item Define each NPC's personality scores, by choosing a random integer between the minimum and maximum value for each of the four personality factors.
    	\item For each of the personality factors, define behavior consisting combinations of animations, movement patterns, facial expressions, body language and speech. Define multiple behaviors for the extremes of each personality factor (IE. 0 percent extraversion and 100 percent extraversion) and preferably also some in-between behaviors. Define at what end of the spectrum each behavior lies. Note that you might have to define different behaviors for the same personality factors: Behavior for use in events and behavior for use when there are no events happening near the NPC.%LOT OF WORK
    	\item Define a series of events that will be triggered throughout the duration of the game. For each of these events, take all of the four personality factors and give them a threshold value ranging from 0 to 100. The higher this threshold is for a given personality factor, the less likely that an NPC's reaction to this event will be driven by that personality factor.
    \end{enumerate}

    Each frame, there will be a check to see wether the NPC is in the vicinity of an event. If so, the algorithm will do the following:

	%UPDATE THE ALGORITHMS

	\newpage
	\section{Interaction between NPC's}
	In order to create a realistic feeling for the player, there also needs to be interaction between the different NPC's, rather than just single NPC's displaying behaviors. One way to do this would be in the form of mini-events between NPC's that only occur when NPC's get within a certain range of eachother. 

	%MORE NPCs
%FRIENDS?
%MEMORY FOR NPC?
	A very basic event that might happen when two NPC's enter each others' personal space is that one of them might try to begin a conversation, if his personality scores \footnote{Within the Big-5 model, Extraversion vs. Introversion and Emotional Stability vs. Neuroticism are the most likely candidates for this task.} allow him to. Depending on the personality scores of the other NPC, he/she might accept the invitation. During the conversation, the body language, speech patterns and facial expressions of both NPC's will be governed by their relevant personality scores, like we talked about in the previous section.

	Another simple interaction which would only involve 2 NPC's at a time would be navigation: When an NPC navigates a crowd of other NPC's, they might back away from him or hold their   ground, depending once again upon certain of their personality scores. This very simple measure would create a more realistic representation of a crowd or group of actual humans.

We thought about including some slightly more socially unacceptable behavior into the mix, like fights and arguments breaking out between NPC's. However, we have decided to omit this kind of behavior for now because we think it might make it harder for the player to notice NPC personalities.

Group behavior is a subject which we are still looking in to: It might really enhance the realism of the game, but it will provide a great technical challenge and it might obscure individual personalities and behaviors in the same way that the aforementioned socially unacceptable behavior might.


	\newpage
	\section{Demonstrating the NPC's}
	 We intend to demonstrate this project by creating a game, or several games, in which player NPC interaction and NPC personalities will be key concepts. The simplest and probably most usefull kind of demo would be to have the player walk around a group of NPC's and have the player identify the NPC who exhibits a certain personality trait the strongest. This would be quite easy to implement: The only addition that is needed is for us to randomly select a certain trait, then store which NPC has the highest score on that trait. Finally, some controls for the player need to be added and the demo is ready. Because we use an XML file to store the ranges a certain personality trait may exhibit for a given NPC population, we can easily create multiple XML files storing various personality ranges and feed them as input to the game whenever we choose, providing us with an easy way of showcasing different NPC populations.

%interrogation game
A more complicated demonstration game would be to have the player interact one-on-one with an NPC within the context of an interrogation game. A possible way to implement this would be to have a certain NPC being suspected to have committed a certain crime, with the player tasked to ask questions about this crime and decide, based on the NPC's responses, body language and facial expression wether the NPC has committed the crime or not. This would require some extra additions to the vanilla concept, like generating random crimes, their circumstances and the NPC's degree of involvement in the crime. It might also require some more advanced facial animations in order to create a believable 'suspect'. If implemented well enough however, it might provide a very entertaining experience to the players. It might also provide data on wether the NPC's personalities can also shine through and be believable in a one-on-one situation.

%VIP protection
Finally, we might expand on the first kind of demo by including more of a challenge factor. This could be done by giving the player a time limit to find an NPC with a certain trait-extreme or by giving a 'profiler' role to the player. This would involve the player having to pick out an NPC from a crowd who is most likely to be agressive or commit a crime, by carefully examining their behavior. This task can be made more challenging by giving the player only one chance: If they choose the wrong NPC, they lose instantly. This might require a more detailed personality model than our current one, which only measures 5 of the primary personality traits of humans. HEXACO, for example, includes factors like fear of physical pain, anxiety, patience etc. which could all be potential characteristics exhibited by criminals who are about to commit a crime.

Another variation on this 'spot the criminal'  idea could be having the character protect a VIP NPC; The player then has to pick out the non-VIP NPC most likely to harm the VIP, based on their behavior. 

All of these examples can provide an entertaining game with the ability to tell us how well the characters emulate real personalities; the amount of time and resources we will have available when we have established a good baseline for procedural avatar personality generation will influence which demoes we will and will not implement.

We want to create each of these demoes as a game for the experience room, in order to provide a more lifelike interaction between the player and the NPC's, and in order to provide data on how well the concept of characters with distinct personalities works within the context of a mixed reality environment. 





  	%Hoe gaat de demo eruitzien, welke events kiezen we uit
  	%Big5 model uitwerking als bijlage toevoegen
>>>>>>> eb341849912d01c83266193940db1db6dab3c6b8

\end{document}